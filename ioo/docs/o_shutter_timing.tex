\documentclass[10pt,a4paper]{article}
\pagestyle{plain}
\textwidth 16cm
\textheight 21cm
\oddsidemargin -0.5cm
\topmargin 0cm

\title{O Shutter Timing}
\author{C. J. Mottram}
\date{}
\begin{document}
\pagenumbering{arabic}
\thispagestyle{empty}
\maketitle
\begin{abstract}
This document describes the model and implemetation used for the O shutter.
\end{abstract}

\centerline{\Large History}
\begin{center}
\begin{tabular}{|l|l|l|p{15em}|}
\hline
{\bf Version} & {\bf Author} & {\bf Date} & {\bf Notes} \\
\hline
0.1 & C. J. Mottram & 12/09/11 & First draft \\
\hline
\end{tabular}
\end{center}

\newpage
\tableofcontents
\listoffigures
\listoftables
\newpage

\section{Introduction}
This document describes the IO:O shutter timing, and how we model it in software. This is important for
O, as the shutter is particularily slow to open and close.

The shutter is a Uniblitz CS90. The shutter specifications \cite{bib:uniblitzcs90spec} contain timing data for the shutter.

\section{Timing Diagram}

The timing diagram is shown in Figure \ref{fig:oshuttim}.

\setlength{\unitlength}{1in}
\begin{figure}[!h]
	\begin{center}
		\begin{picture}(10.0,5.5)(0.0,0.0)
			\put(0,0){\special{psfile=o_shutter_delays.eps   hscale=75 vscale=75}}
		\end{picture}
	\end{center}
	\caption{\em O Shutter Timing Diagram.}
	\label{fig:oshuttim} 
\end{figure}

The user specifies the exposure length {\bf Texp}. The user expects the {\bf UTSTART} FITS header keyword to record
when the shutter started to open. We have decided the exposure length {\bf Texp} is defined for the centre of the CCD
where a point source is likely to be placed. The edges of the CCD will end up with a shorter exposure length 
{\bf $Texp - SOD - SCD$}.

When the software commands the shutter to open (the timing board DSP code) the relevant utility board signal line is set high. There is then a {\bf Shutter Trigger Delay} {\bf STD} before the shutter control electronics start to open the 
shutter. The shutter takes a long time to open (it is a mechanical device) called the {\bf Shutter Open Delay} {\bf SOD}. 

After the relevant utility board signal line is set high the utility board should wait {\bf $STD + Texp - SCD$} before
setting the utility board signal line low again. The shutter again takes a while to close (the {\bf Shutter Close Delay} {\bf SCD}), and then the utility board waits for a few more milliseconds (the {\bf Readout Delay} {\bf RD}) before actually starting to read out the CCD.

\section{Uniblitz Timings}

The Uniblitz CS90 timings, as taken from the \cite{bib:uniblitzcs90spec}, are in Table \ref{tab:uniblitzcs90time}:

\begin{table}[!h]
\begin{center}
\begin{tabular}{|l|l|p{15em}|l|}
\hline
{\bf Timing Diagram} & {\bf Uniblitz Diagram} & {\bf Uniblitz Description} & {\bf Value} \\ \hline
STD                  & O-A		      & Delay time on opening after current is applied & 20 ms \\ 
SOD                  & A-C                    & Transfer time on opening                       & 70 ms \\ 
SCD                  & E-G                    & Transfer time on closing                       & 90 ms \\ \hline
\end{tabular}
\end{center}
\caption{\em Uniblitz CS90 timings.}
\label{tab:uniblitzcs90time}
\end{table}

\section{Timing board code}

\subsection{As supplied}

The supplied timing board DSP code for exposures starts at the {\bf START\_EXPOSURE} routine. This works as follows:

\begin{itemize}
\item {\bf IIA} is sent to the PCI board to initialise the PCI image address 
      (i.e. clear the PCI DMA buffer for the image readout).
\item The {\bf CLR\_CCD} sub-routine is called to clock out residual charge on the CCD.
\item The {\bf TST\_RCV} routine address is put into the {\bf IDL\_ADR} memory location, so that commands can be
      processed during an exposure (abort commands).
\item The {\bf SERIALS\_EXPOSE} waveform is clocked to but the serial clocks in the right state for exposure. 
      The {\bf WAIT\_TO\_FINISH\_CLOCKING} is called to ensure the clocking has been completed.
\item If the {\bf SHUT} bit is set in {\bf X:STATUS}, the {\bf OSHUT} sub-routine is called to open the shutter.
\item The return address for the {\bf EXPOSE} routine is setup, and then {\bf EXPOSE} is called to wait for the
      exposure length, whilst checking for commands over the optical fibre.
\item The {\bf PCI\_READ\_IMAGE} routine is called to put the PCI card in a state to receive image data. The {\bf ST\_RDC} bit is set in {\bf X:STATUS}.
\item If the {\bf SHUT} bit is set in {\bf X:STATUS}, the {\bf CSHUT} sub-routine is called to close the shutter.
\item A loop is entered to delay for {\bf Y:SHDEL} milliseconds.
\item {\bf RDCCD} is called to readout the CCD.
\end{itemize}

\subsection{Clearing the CCD}

As supplied, the SDSU code calls the {\bf CLR\_CCD} routine before opening the shutter and starting the exposure.
For RATCam, we commented this out of the SDSU timing board code, and manually issued a {\bf CLR} command from the
C layer software a few seconds ($\sim$ 10) before the exposure started. This allowed us to know the exposure
start time and hence UTSTART with more accuracy - comments in the RATCam code suggest {\bf CLR\_CCD} took $\sim$ 5
seconds.

The {\bf CLR\_CCD} routine in IO:O seems to be a lot quicker, but the pulse widths are shorter than spec so we 
are currently unsure this will clock all the accumulated charge out correctly. We could comment out the 
{\bf CLR\_CCD} routine from the DSP code and call the {\bf CLR} command from the C layer software a few seconds before.
Or we could measure the length of time the {\bf CLR\_CCD} routine takes and take this into account when issuing
a {\bf SEX} command (for a RUNAT command, and for calculating {\bf UTSTART}).


\subsection{Suggested changes}

All the above requires the following changes to be made.

The exposure length is passed to the DSP code using the {\bf SET} command. The {\bf EXPOSE} routine starts a timer
that waits until that time has elapsed. This exposure length should be modified as follows:
${\bf Texp} = {\bf STD} + {\bf Texp} - {\bf SCD}$. This can be done in the C layer.

The value of {\bf UTSTART} should be the time the exposure is started (when the {\bf SEX} command returns?)
 plus {\bf STD}. This can be done in the C layer.

The delay at the end of the {\bf EXPOSE} routine ({\bf Y:SHDEL}) should be set to ${\bf SCD} + {\bf RD}$.
This again can be done from the C layer.

Note for a RUNAT command, where we want to start an exposure at a specific time, we should send the
{\bf SEX} command {\bf STD} milliseconds early. Given the number of milliseconds we are talking about ($\sim$ 20ms)
and the overall speed of the shutter this is not very important, but it could be done in the C layer if necessary anyway.

So it appears the DSP code does not need to change at all, and the changes can be done in the C layer, with
config passed down from the Java layer.

\begin{thebibliography}{99}
\addcontentsline{toc}{section}{Bibliography}

\bibitem{bib:uniblitzcs90spec}{\bf CS90 Shutter Specifications}
Uniblitz \newline{\em http://www.uniblitz.com/resources\_filelibrary/cs90\_10\_16\_08.pdf}

\end{thebibliography}

\end{document}
